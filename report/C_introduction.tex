\section{Introduction}
%(recommended size, including points 1 and 2: 1 page): describe the problem, the existing systems and/or tools about which you know system you are about to implement, and the structure of the remainder of the article. Use one short paragraph for each.
Gaming has become the largest form of entertainment worldwide (beating music and film\footnote{\url{http://www.theguardian.com/technology/gamesblog/2009/sep/27/videogames-hollywood}}) and a large portion of gaming occurs online\footnote{\url{http://www.gamesindustry.biz/articles/2013-05-02-72-percent-of-gamers-play-online-npd}}. Especially multi-player gaming, in which millions of players are interacting with each other. The problem is that if these games are not carefully designed, the whole game can crash when a single or more client(s) or server(s) fails. This is very undesirable when the game is dependent upon thousands of those. This paper focuses on designing a game (Dragon Arena Systems (DAS)) carefully enough, so that the game can continue even after failures of clients or a few servers.\\
In solving this problem, we are not starting from scratch. A basis for the system is a not-yet-compiling, non-distributed code-project that in broad strokes implemented the game on a single machine. Also the paper by Cronin et al. on Trailing State Servers (TSSs) is used. From this paper the concept of TSSs is derived, the idea of implementing DAS using TSSs being ours.\\
For the system itself there is the issue of the difference between the system we wanted to implement and the system we managed to implement. We wanted to implement the system in a distributed way using TSSs. However in practise we couldn't manage to work out the details of that approach in time. Therefore, we implemented a non-distributed version that works and the set-up for a version that works with TSSs on GitHub\footnote{\url{https://github.com/javto/DAS\_distributed/}}. In this report we are going to focus on the ideal version with TSSs. \\
After this introduction we will give a more extensive description of the DAS application and its requirements in Section 2. Then in Section 3 an overview of the systems itself and its features is shown (either implemented or desired in the TSSs version). In Section 4, the design of a number of experiments are elucidated. Finally, a discussion on the trade-offs made in the system and a conclusion are found in Sections 5 and 6 respectively.