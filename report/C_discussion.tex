\section{Discussion}
\label{chap:discussion}
% (recommended size: 1 page): summarize the main findings of your work and discuss the tradeoffs inherent in the design of the DAS system. Should WantGame use a distributed system to implement the DAS system? Try to extrapolate from the results reported in Section 6.b for system workloads that are orders of magnitude higher than what you have tried in real-world experiments.
This section describes the time it will it take to implement the proposed system,  and the trade-off between centralized and distributed manner is given as well as a recommendation on how to proceed.
To make a prototype of the proposed plan it will take at least 50 man hours of work. After this, all the bugs, edge-cases and other problems should be fixed before the experiments in Section \ref{chap:expresults} can be performed. This will take a lot of time still, before a meaningful conclusion can be reached to recommend WantGame about their desire of a distributed game. 
At this point, the distributed game brings a lot more difficulty than a centralized game, so if WantGame needs a short time-to-market, a centralized version would make sense to start generating revenue, possibly with servers from Amazon\footnote{\url{http://aws.amazon.com/ec2/}} for a very easy and robust start-up. In the meanwhile, the distributed version can be further developed. This way, WantGame gets a sense of interest from the players and thus see, if a scalable game is necessary. The disadvantage here is that the game data like statistics and other player information has to be transported to the new distributed system. One way to overcome this is by making the distributed way backwards compatible with the old data and by phasing out the central servers gradually so information gets copied to the new system.