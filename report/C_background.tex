\section{Background on Application}
\label{chap:background}
%(recommended size: 0.5 pages): describe the DAS application (1 paragraph) and its requirements (1 paragraph per each of consistency, scalability, fault-tolerance, and performance).
Dragon Arena Systems (DAS, as specified by WantGame BV) is game based on a board consisting of 25 by 25 squares. Squares can be occupied by either dragons or players. Players can move, heal other players and attack dragons, with each action having a limited range and power. Dragons can only attack players, also within a limited range and with a limited attack power. Players can join or leave the game and the game finishes when there are either no more players or no more dragons.\\
The first of our requirements is that the game needs to operate as specified above. To test the system, we simulate the behaviour of the players in the following way: if a nearby player has less than 50\% health, heal that player. Otherwise strike a dragon if there is one in range. If neither is in range, go towards the closest dragon. For the behaviour of clients of connecting and disconnecting to the server, we must use real data from the Game Trace Archive.\\
Next we require fault tolerance in our system. This means that it must be possible for at least 1 client or server to crash at a time, while the game keeps running. Of course there might be some effects of a crash on the game (most notably the crash of a client will remove a player from the game). However these effects need to be minimal and non-disruptive in so far as that the (other) players need to be able to continue playing.\\
Our last requirement is that the system needs to be scalable. In this case scalable is defined as follows: the system must be able to operate with: 100 players, 20 dragons and 5 servers. Operate in this case means: run on the DAS-computer service without crashing until either only players or only dragons are left.