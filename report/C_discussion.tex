\section{Discussion}
\label{chap:discussion}
% (recommended size: 1 page): summarize the main findings of your work and discuss the tradeoffs inherent in the design of the DAS system. Should WantGame use a distributed system to implement the DAS system? Try to extrapolate from the results reported in Section 6.b for system workloads that are orders of magnitude higher than what you have tried in real-world experiments.
This section describes what the experiment results could mean for the system. Next, it elucidates the time it will it take to implement the proposed system. Furthermore, the trade-off between centralized and distributed manner is given as well as a recommendation on how to proceed.
The experiment designs in Section \ref{chap:expresults} will give insight in to how well the system scores on a number of different characteristics of a distributed system. Moreover, it gives information about how to improve the system. For the requirements, it said that five servers were needed to show the system, but in the real world a lot more players could be connected at which point more servers need to be added. This raises the question about when the system should boot a new server, and perhaps when a lot of players disconnect, when should servers be shut down. The threshold for booting a server can be calculated by estimating the number of players that will arrive in the upcoming time and by looking at the average start up time of a server. If in the time of starting up a server, other servers would be on their limit, looking at the estimation of players arriving, and the traffic can not be redirected, a new server should be started. The safety margin here is dependent on the error in the calculation of the estimation of player arrival and the variance of start-up times for servers. A server will be shut down when there are no more players connected to it and the overall load of the system, within the time it will take to boot up a new server, will not exceed server capacities. To keep the utilisation of every server as high as possible without introducing lag, we want to redirect players connected to a relatively idle server to more used servers so the idle server can be shut down. This will only happen when the estimation of player arrival will not exceed server capacities in the system without the idle server. This load balancing will not be beneficial in flash crowd scenarios, therefore we propose to make the estimation based on the early history at hand and information about  yearly events and other known busy periods. This way, it will quickly detect these scenarios and boot up new servers. To make the system more robust again (D)DoS attacks, and flash crowd scenarios we could implement either more safety margin in server thresholds in terms of booting up new servers or offload unforeseen heavy loads to cloud services such as the elastic cloud from Amazon: EC2\footnote{\url{http://aws.amazon.com/ec2/}}.
To make a prototype of the proposed plan it will take at least 50 man hours of work. After this, all the bugs, edge-cases and other problems should be fixed before the experiments in Section \ref{chap:expresults} can be performed. This will take a lot of time still, before a meaningful conclusion can be reached to recommend WantGame about their desire of a distributed game. 
At this point, the distributed game brings a lot more difficulty than a centralized game, so if WantGame needs a short time-to-market, a centralized version would make sense to start generating revenue, possibly with servers from Amazon for a very easy and robust start-up. In the meanwhile, the distributed version can be further developed. This way, WantGame gets a sense of interest from the players and thus see, if a scalable game is necessary. The disadvantage here is that the game data like statistics and other player information has to be transported to the new distributed system. One way to overcome this is by making the distributed way backwards compatible with the old data and by phasing out the central servers gradually so information gets copied to the new system.